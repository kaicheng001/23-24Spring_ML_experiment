\documentclass[12pt]{article}
\usepackage[UTF8]{ctex}
\usepackage[margin=1in]{geometry}
\usepackage{graphicx} % For including figures
\usepackage{amsmath}  % For math fonts, symbols and environments
\usepackage{pgfgantt} % For Gantt charts
\usepackage[hidelinks]{hyperref} % For hyperlinks
\usepackage{enumitem} % For customizing lists
\definecolor{blue}{HTML}{74BBC9}
\definecolor{yellow}{HTML}{F7E967}
\usepackage{listings}
\usepackage{xcolor}
\usepackage{tocloft} % 导入tocloft包
\usepackage{zi4}
\usepackage{fontspec}

\usepackage{graphicx} % For including graphics
\usepackage{booktabs} % For professional looking tables
\usepackage{array}    % For extended column definitions
\usepackage{amsmath}  % For math environments like 'equation'
\usepackage{amsfonts} % For math fonts like '\mathbb{}'
\usepackage{amssymb}  % For math symbols
\usepackage{caption}  % For custom captions
\usepackage[table]{xcolor} % For coloring tables

% Set page margins
\usepackage[margin=1in]{geometry}

% 目录标题样式定义
\renewcommand{\cfttoctitlefont}{\hfill\Large\bfseries}
\renewcommand{\cftaftertoctitle}{\hfill\mbox{}\par}

% Monokai theme with a lighter background
\definecolor{codegreen}{rgb}{0,0.6,0}
\definecolor{codegray}{rgb}{0.5,0.5,0.5}
\definecolor{codepurple}{rgb}{0.58,0,0.82}
\definecolor{backcolour}{rgb}{0.95,0.95,0.92}
\setmonofont{Source Code Pro}[Contextuals={Alternate}]

\lstdefinestyle{mystyle}{
    backgroundcolor=\color{backcolour},   
    commentstyle=\color{codegreen},
    keywordstyle=\color{magenta},
    numberstyle=\tiny\color{codegray},
    stringstyle=\color{codepurple},
    basicstyle=\ttfamily\footnotesize,
    breakatwhitespace=false,         
    breaklines=true,                 
    captionpos=b,                    
    keepspaces=true,                 
    numbers=left,                    
    numbersep=5pt,                  
    showspaces=false,                
    showstringspaces=false,
    showtabs=false,                  
    tabsize=2
}

\lstset{style=mystyle}




\title{\textbf{ML第二次实验报告}}
\author{58122204 谢兴}
\date{\today}

\begin{document}


\begin{titlepage}
  \centering
  \vspace*{60pt}
  \Huge\textbf{机器学习实验报告}

  \vspace{100pt}
  \Large
  实验名称:建立全连接神经网络

  \vspace{25pt}
  学生:谢兴

  \vspace{25pt}
  学号:58122204

  \vspace{25pt}
  日期:2024/4/25

  \vspace{25pt}
  指导老师:刘胥影

  \vspace{25pt}
  助教:田秋雨



\end{titlepage}


\newpage
\tableofcontents


\section{任务描述}
实现卷积神经网络CNN,并使用CIFAR-10数据集进行进行图片分类任务。

CIFAR-10是计算机视觉领域中的一个重要的数据集 。 原始数据集分为训练集和测试集,
其中训练集包含50000张、测试集包含300000张图像。 在测试集中,10000张图像将被用于评估,
而剩下的290000张图像将不会被进行评估,包含它们只是为了防止手动标记测试集并提交标记结果。
这些图片共涵盖10个类别:飞机、汽车、鸟类、猫、鹿、狗、青蛙、马、船和卡车,高度和宽度均为32像素并有三个颜色通道(RGB)。
图1的左上角显示了数据集中飞机、汽车和鸟类的一些图像。本实验使用部分的CIFAR-10数据集,其中训练集共包含5000张png格式的图像,
每个类别包含500张;测试集共包含5000张jpg格式的图像。


\section{实验要求}

在掌握CNN原理的基础上,使用CNN在CIFAR-10数据集上进行图片分类任务。在此实验中,要求掌握以下内容:

\begin{enumerate}
  \item 图片数据的加载和预处理,熟悉PyTorch中对数据集的处理
  \item 使用PyTorch实现CNN网络架构
  \item 掌握深度学习模型的训练过程,正确地进行tensor的运算
  \item 计算损失函数和准确率,对模型性能进行评估
  \item 调整参数设置,进行参数分析实验
\end{enumerate}




\section{模型架构}





\section{CNN模型原理阐述}
\section{CIFAR-10数据集的内容和数据预处理}

\section{CNN模型构建}

\section{实验设置}


\section{实验结果}


\section{参数分析实验}



\newpage
% 开始附录部分
\appendix
% 手动添加附录到目录中
\addcontentsline{toc}{section}{附录}
\section{使用Python编程构建手动实现单隐层全连接神经网络Code}


\end{document}
