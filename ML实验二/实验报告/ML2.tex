\documentclass[12pt]{article}
\usepackage[UTF8]{ctex}
\usepackage[margin=1in]{geometry}
\usepackage{graphicx} % For including figures
\usepackage{amsmath}  % For math fonts, symbols and environments
\usepackage{pgfgantt} % For Gantt charts
\usepackage[hidelinks]{hyperref} % For hyperlinks
\usepackage{enumitem} % For customizing lists
\definecolor{blue}{HTML}{74BBC9}
\definecolor{yellow}{HTML}{F7E967}
\usepackage{listings}
\usepackage{xcolor}
\usepackage{tocloft} % 导入tocloft包
\usepackage{zi4}
\usepackage{fontspec}

% 目录标题样式定义
\renewcommand{\cfttoctitlefont}{\hfill\Large\bfseries}
\renewcommand{\cftaftertoctitle}{\hfill\mbox{}\par}

% Monokai theme with a lighter background
\definecolor{codegreen}{rgb}{0,0.6,0}
\definecolor{codegray}{rgb}{0.5,0.5,0.5}
\definecolor{codepurple}{rgb}{0.58,0,0.82}
\definecolor{backcolour}{rgb}{0.95,0.95,0.92}
\setmonofont{Source Code Pro}[Contextuals={Alternate}]

\lstdefinestyle{mystyle}{
    backgroundcolor=\color{backcolour},   
    commentstyle=\color{codegreen},
    keywordstyle=\color{magenta},
    numberstyle=\tiny\color{codegray},
    stringstyle=\color{codepurple},
    basicstyle=\ttfamily\footnotesize,
    breakatwhitespace=false,         
    breaklines=true,                 
    captionpos=b,                    
    keepspaces=true,                 
    numbers=left,                    
    numbersep=5pt,                  
    showspaces=false,                
    showstringspaces=false,
    showtabs=false,                  
    tabsize=2
}

\lstset{style=mystyle}




\title{\textbf{ML第二次实验报告}}
\author{58122204 谢兴}
\date{\today}

\begin{document}


\begin{titlepage}
    \centering
    \vspace*{60pt}
    \Huge\textbf{机器学习实验报告}

    \vspace{100pt}
    \Large
    实验名称:建立全连接神经网络

    \vspace{25pt}
    学生:谢兴

    \vspace{25pt}
    学号:58122304

    \vspace{25pt}
    日期:2024/4/22

    \vspace{25pt}
    指导老师:刘胥影

    \vspace{25pt}
    助教:田秋雨



\end{titlepage}


\newpage
\tableofcontents
	

\section{任务描述}
通过两种方式实现全连接神经网络,并对图片分类任务行进行测试与实验。
\begin{enumerate}
    \item 手动实现简单的全连接神经网络
    \item 使用Pytorch库简洁实现全连接神经网络
\end{enumerate}

Fashion-MNIST图片分类数据集包含10个类别的时装图像,训练集有60,000张图片,测试集中有10,000张图片。图片为灰度图片,高度(h)和宽度(w)均为28像素,通道数(channel)为1。10个类别分别为:t-shirt(T恤), trouser(裤子), pullover(套衫), dress(连衣裙), coat(外套), sandal(凉鞋),shirt(衬衫),sneaker(运动鞋),bag(包),ankle boot(短靴)。使用训练集数据进行训练,测试集数据进行测试。
\section{教学要求}
\begin{enumerate}
    \item 掌握多层前馈神经网络及BP算法的原理与构建
    \item 了解PyTorch库,掌握本实验涉及的相关部分
    \item 进行参数分析实验,理解学习率等参数的影响
\end{enumerate}

\section{实验要求}
\subsection{\textbf{使用Python编程构建手动实现单隐层全连接神经网络}}

\textbf{模型架构}
\begin{itemize}
    \item 输入层 \( 28 \times 28 = 784 \) 个节点,输出层10个节点,隐藏层256个节点。注意,可以将这两个变量都视为超参数。通常选择2的若干次幂作为层的宽度。因为内存在硬件中的分配和寻址方式,这么做往往可以在计算上更高效。
    \item 激活函数:ReLU函数
    \item 损失函数:Cross entropy
    \item 性能指标:准确率
    \item 优化算法:实现标准BP或小批量梯度下降算法均可
\end{itemize}

\textbf{实现内容}
\begin{enumerate}
    \item 初始化模型参数:对于每一层都要记录一个权重矩阵和一个偏置向量。
    \item 设置激活函数:使用ReLU函数作为激活函数,要求手动实现该函数。
    \item 前向计算:实现该函数。注意:需要将每个二维图像转化为向量进行操作。
    \item 设置损失函数:使用cross entropy作为损失函数。可以自己手动实现,也可以直接调用\texttt{nn.CrossEntropyLoss}函数。
    \item 训练模型:
    \begin{enumerate}
        \item 实现训练函数:该训练函数将会运行多个迭代周期(由\textit{num\_epochs}指定)。在每个迭代周期结束时,利用\textit{test\_iter}访问到的测试数据集对模型进行评估。利用后面给出的Animator类来可视化训练进度。
        \item 可以使用PyTorch内置的优化器(\texttt{torch.optim.SGD}),也可以使用自己定制的优化器。
        \item 可以调用\texttt{torch.optim.SGD}函数进行参数更新。
        \item 迭代周期数\textit{epoch}设置为10,学习率设置为0.1,训练模型。
    \end{enumerate}
    \item 设置性能函数:使用准确率\textit{accuracy}作为性能指标。实现该函数。
    \item 模型评估:
    \begin{enumerate}
        \item 对测试集数据进行测试。
        \item 进行性能评估。
    \end{enumerate}
    \item 参数分析实验:
    \begin{enumerate}
        \item 在所有其他参数保持不变的情况下,更改超参数\textit{num\_hiddens}的值,并查看此超参数的变化对结果有何影响。确定此超参数的最佳值。
        \item 改变学习速率会如何影响结果?保持模型架构和其他超参数(包括轮数)不变,学习率设置为多少会带来最好的结果?
    \end{enumerate}
\end{enumerate}

\subsection{\textbf{使用PyTorch库简洁实现全连接神经网络}}

手动实现一个简单的多层神经网络是很容易的。然而如果网络有很多层,从零开始实现会变得很麻烦。可以使用高级API如PyTorch库简洁实现。


\begin{enumerate}
    \item 请使用PyTorch库简洁实现前述的全连接神经网络,并进行模型评估。
   \begin{enumerate}
    \item 优化器:使用\texttt{torch.optim.SGD}
    \item 小批量数据载入函数参见提供的代码。
   \end{enumerate}
    \item \textbf{参数分析实验}
    \begin{enumerate}
        \item 尝试添加不同数量的隐藏层(也可以修改学习率),怎么样设置效果最好?
        \item 尝试不同的激活函数,哪个效果最好?
        \item 尝试不同的方案来初始化权重,什么方法效果最好?
    \end{enumerate}
\end{enumerate}

\subsection{\textbf{提交要求}}
其中报告内容包括以下几个部分:
\begin{enumerate}
    \item 手动实现单隐层全连接神经网络
    \begin{enumerate}
        \item 训练过程中,训练集与验证集误差随epoch变化的曲线图
        \item 性能评估结果
        \item 参数分析实验:包括实验设置与结果分析
    \end{enumerate}
    \item 使用PyTorch库简洁实现全连接神经网络
    \begin{enumerate}
        \item 性能评估结果
        \item 参数分析实验:包括实验设置与结果分析
    \end{enumerate}
\end{enumerate}

\section{数据集}

\section{训练与测试}

\section{实验总结}



\newpage
% 开始附录部分
\appendix
% 手动添加附录到目录中
\addcontentsline{toc}{section}{附录A}
\section*{附录A}
% 附录内容...
\subsection*{附录A1}
% ... 附录内容 ...
\begin{lstlisting}[language=Python, caption=Python example]
import pandas as pd
import numpy as np
import seaborn as sns
import pickle
from sklearn.preprocessing import MinMaxScaler
from statsmodels.stats.outliers_influence import
variance_inflation_factor
from sklearn.decomposition import PCA
from sklearn.model_selection import train_test_split
from sklearn.preprocessing import MinMaxScaler
from sklearn.decomposition import PCA
from statsmodels.stats.outliers_influence import
variance_inflation_factor
import matplotlib.pyplot as plt
from sklearn.linear_model import LinearRegression
from sklearn.metrics import r2_score, mean_squared_error
# 读取 CSV 文件
df = pd.read_csv('song_data.csv')
# # 标记重复行(完全相同的行视为重复)
# df['is_duplicate'] = df.duplicated(keep=False) # keep=False 会标记所
有的重复项
#
# # 计算重复与非重复的行数
# duplicate_distribution = df['is_duplicate'].value_counts()
#
# # 绘制重复行分布的条形图
# plt.figure(figsize=(8, 6))
# duplicate_distribution.plot(kind='bar')
# plt.title('Distribution of Duplicate Rows')
# plt.xticks(ticks=[0, 1], labels=['Unique Rows', 'Duplicate Rows'],
rotation=0)
# plt.ylabel('Count')
# plt.xlabel('Row Type')
# plt.show()
\end{lstlisting}

\addcontentsline{toc}{section}{附录B}
\section*{附录B}
% 另一个附录的内容...
\subsection*{附录B1}
% ... 附录内容 ...



\end{document}
